\documentclass{beamer}
\usetheme{CambridgeUS}

\title{Assignment 13: Papoulis Chapter 11 }
\author{Dhatri Reddy}
\date{\today}
\logo{\large \LaTeX{}}

\usepackage{amsmath}
\usepackage{romannum}
\usepackage{enumitem}
\setbeamertemplate{caption}[numbered]{}
\providecommand{\pr}[1]{\ensuremath{\Pr\left(#1\right)}}
\providecommand{\cbrak}[1]{\ensuremath{\left\{#1\right\}}}
\providecommand{\brak}[1]{\ensuremath{\left(#1\right)}}


\begin{document}

\begin{frame}
    \titlepage 
\end{frame}

\logo{}

\begin{frame}{Outline}
    \tableofcontents
\end{frame}

\section{Question}
\begin{frame}{Question}
    \begin{block}{Problem 11.4}
    The process x\brak{t} is WSS and 
    $y''\brak{t} + 3y'\brak{t} + 2y\brak{t} = x\brak{t}$
    Show that
    \begin{enumerate}[label=(\alph*)]
    \item $R''_{yx}\brak{\tau} + 3R'_{yx}\brak{\tau} + 2R_{yx}\brak{\tau} = R_{xx}\brak{\tau} $\\
    $R''_{yy}\brak{\tau} + 3R'_{yy}\brak{\tau} + 2R_{yy}\brak{\tau} = R_{xy}\brak{\tau} $ all $\tau$
    \item If $R_{yx}\brak{\tau} = q\delta\brak{\tau}$, then $R_{yx}\brak{\tau} = 0$ for $\tau < 0$ and for $\tau > 0$:\\
    $R''_{yx}\brak{\tau} + 3R'_{yx}\brak{\tau} + 2R_{yx}\brak{\tau} = 0 $  $R_{yx}\brak{0} = 0$  $R'_{yx}\brak{0^{+}} = q$ \\
    $R''_{yy}\brak{\tau} + 3R'_{yy}\brak{\tau} + 2R_{yy}\brak{\tau} = 0 $  $R_{yy}\brak{0} = \frac{q}{12}$ $R'_{yy}\brak{0} = 0$
    \end{enumerate}
\end{block}
\end{frame}

\section{Solving a}
\begin{frame}
\frametitle{Solving a}
By multiplying both sides of the equation given in question by x\brak{t-\tau} and y\brak{t+\tau} we can conclude that\\
$R''_{yx}\brak{\tau} + 3R'_{yx}\brak{\tau} + 2R_{yx}\brak{\tau} = R_{xx}\brak{\tau} $\\
$R''_{yy}\brak{\tau} + 3R'_{yy}\brak{\tau} + 2R_{yy}\brak{\tau} = R_{xy}\brak{\tau} $

\end{frame}

\section{Solving b}
\begin{frame}
\frametitle{Solving b}

From \brak{a} it follows that 
$R''_{yx}\brak{\tau} + 3R'_{yx}\brak{\tau} + 2R_{yx}\brak{\tau} = q\delta\brak{\tau}$\\
Since $R_{xx}\brak{\tau} = 0$ for $\tau < 0$, the above shows that 
$R_{yx}\brak{\tau} = 0$ for $\tau < 0^{-}$ $R'_{yx}\brak{0^{-}} = 0$

\begin{align}
    S_{yx}\brak{s} = \frac{q}{s^{2} + 3s + 2}\\
R_{yx}\brak{0^{+}} = \lim\limits_{s \to \infty} s S_{yx}\brak{s} = q\\
R'_{yx}\brak{0^{+}} = \lim\limits_{s \to \infty} s^{2} S_{yx}\brak{s} = 0
\end{align}

$R''_{yy}\brak{\tau} + 3R'_{yy}\brak{\tau} + 2R_{yy}\brak{\tau} = R_{xy}\brak{\tau} = R_{xy}\brak{-\tau} = 0$ for $\tau > 0$

\end{frame}

\section{Solving b}
\begin{frame}
\frametitle{Solving b}
\begin{align}
    S_{yy}\brak{s} = \frac{q}{\brak{s^{2}+ 3s + 2}\brak{s^{2} - 3s + 2}}\\
    = \frac{\frac{qs}{12} + \frac{q}{4}}{s^{2} + 3s + 2} + \frac{-\frac{qs}{12} + \frac{q}{4}}{s^{2} - 3s + 2}\\
    S^{+}_{yy}\brak{s} = \frac{\frac{qs}{12} + \frac{q}{4}}{s^{2} + 3s + 2}\\
    R^{+}_{yy}\brak{0^{+}} = R_{yy}\brak{0} = \lim\limits_{s \to \infty} s^{2} S^{+}_{yy}\brak{s} =  \frac{q}{12}\\
    R'_{yy}\brak{0} = \lim\limits_{s \to \infty} s \brak{s S^{+}_{yy}\brak{s} - \frac{q}{12}} = 0 
\end{align}
\end{frame}
\end{document}